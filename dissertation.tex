% example for dissertation.sty
\documentclass[
% Replace oneside by twoside if you are printing your thesis on both sides
% of the paper, leave as is for single sided prints or for viewing on screen.
oneside,
%twoside,
11pt, a4paper,
footinclude=true,
headinclude=true,
cleardoublepage=empty
]{scrbook}

\usepackage{comment}
\usepackage{dissertation}
\usepackage{braket}
\usepackage{mathrsfs}
\newtheorem{definition}{Definition}[section]
\usepackage{mathtools}
\usepackage{amssymb}
\usepackage{indentfirst} 
\usepackage{tikz}
\usepackage{physics}
%usetikzlibrary{arrows}
\DeclarePairedDelimiter\ceil{\lceil}{\rceil}
\DeclarePairedDelimiter\floor{\lfloor}{\rfloor}
\usepackage{subfiles}
\usepackage{qcircuit}
\usepackage{subcaption}
\usepackage{graphicx}
% ----------------------------------------------------------------
%University %(untomment if you need to change default values)
\gdef\school{Escola de Engenharia (ainda a ver)}
\gdef\department{Física}
\gdef\university{Uni}
\gdef\masterdegree{Física da informação.}

% Title
\titleA{Staggered Quantum Walks in Qiskit??}
\titleB{Second Part of Title} % (if any)
\subtitleA{First Part of Subtitle}
\subtitleB{Second part of Subtitle} % (if any)
\usetikzlibrary{arrows}


% Author
\author{Jaime Santos}

% Supervisor(s)
\supervisor{Luís Barbosa}
\cosupervisor{Bruno Chagas}


% Date
\date{\myear} % change to text if date is not today

% Keywords
%\keywords{master thesis}

% Glossaries & Acronyms
%\makeglossaries  %  either use this ...
%\makeindex	   % ... or this

% Define Acronyms
%\input{sec/acronyms}
%\glsaddall[types={\acronymtype}]


\ummetadata % add metadata to the document (author, publisher, ...)

\begin{document}
% Cover page ---------------------------------------
\umfrontcover	
\umtitlepage

% Add acknowledgements ----------------------------
\chapter*{Acknowledgements}
Write acknowledgements here


% Add abstracts (en,pt) ---------------------------
\chapter*{Abstract}
Implementation of staggered quantum walks in qiskit.

\cleardoublepage
\chapter*{Resumo}
Pensar no que escrever aqui.
Hello!	

% Summary Lists ------------------------------------
\tableofcontents
\listoffigures
\listoftables
%\lstlistoflistings
%\listofabbreviations


\pagenumbering{arabic}

% CHAPTER - Introduction -------------------------
\chapter{Introduction}

\section{Brief history of quantum computing}
\subfile{Chapters/Chapter1/historyQC}
% \subsection{[TEMP ARTIGOS RELEVANTES]}
%     kendon2006 - Introducao historia CQ - Pasta introqw/Coin
% 	toffoli1980- Reversible Computing - Pasta intromain \\
% 	gottesman1998- The Heisenberg Representation of Quantum Computers - Pasta intromain\\
% 	Random walks classicas - %https://pt.wikipedia.org/wiki/Passeio_aleat%C3%B3rio 
\section{Classical and Quantum Random Walks}
\subfile{Chapters/Chapter1/randomWalks}
%  Balu2017 introduçao\\
%  Quantum walks are typically classified into discrete- and
%         continuous-time quantum walks with the main difference
%         being whether free evolution is interrupted by quantum
%         coin ‘flips’ followed by coin-dependent translations or
%         whether evolution is continuous involving entangling between the walker and internal, or coin, degrees of freedom [42] \textbf{topologicalwalks.pdf}
\section{State of the Art quantum walks implementations}
%  implementacao em circuitos
%  Final da seccao - implementacao em quantum computers. nao ha muito na literatura.
%  Mencionar que o proximo capitulo tem implementacoes de outros algoritmos.\\
%  timeline da wikipedia\\
%  ler artigos das implementaçoes circuitos locke a e b; douglas wang\\
%  Quantum walks have become germane to quantum
%         computation [20–25] and quantum simulation [26–29]
%         and single-walker versions are amenable to experimental implementations including ion traps [30, 31], superconducting systems [32, 33], nuclear magnetic resonance [34, 35], optical lattice [36, 37], and both freespace linear optics [38, 39] and on photonic chips [40, 41] \textbf{topologicalwalks.pdf}
\section{Text overview and contributions}

\chapter{Quantum Computing}
\section{Mathematical foundations QM}
%TODO: postulados aqui
\section{Quantum Fourier Transform}\label{sec:chapQFT}
\subfile{Chapters/Chapter2/qFourierT}

\chapter{Quantum Walks and searching problems}\label{chap:chap3}
\section{Classical Random Walk}\label{sec:chap3ClassicalWalk}
\subfile{Chapters/Chapter3/classicalWalk}
\section{Coined Quantum Walk}\label{sec:chap3Coinedwalk}
\subfile{Chapters/Chapter3/coinedQuantumWalk}
\section{Continuous-Time Quantum Walk}\label{sec:chap3Contwalk}
\subfile{Chapters/Chapter3/continuousQuantumWalk}
\section{Staggered Quantum Walk}\label{sec:chap3StagWalk}
\subfile{Chapters/Chapter3/staggeredQuantumWalk}
\section{Grover's algorithm}
\subfile{Chapters/Chapter3/groverAlgorithm}
\section{Search problems with Quantum Walks}\label{sec:chap3Search}
\subfile{Chapters/Chapter3/searchProblemsQuantumWalk}

% CHAPTER - Application -------------------------
\chapter{Implementations and Applications}
%TODO:mencionar PC IBMQ
%TODO: Phase oracles onde implementa um operador unitario diagonal com as fases. Explorar como e que eles implementam o diag.
%TODO: Phase kickback?
%\section{Phase Oracle} 
	%% https://qiskit.org/documentation/stubs/qiskit.circuit.library.Diagonal.html
	%% https://arxiv.org/pdf/quant-ph/0406176.pdf
	%% Fazer o diag() sem otimizacao.
%\section{Phase kickback}
%\section{Grover}
%TODO: O difusor tambem tem um phase oracle.
\section{Coined}\label{coinedQWQiskit}
%TODO: 1. Recapitular a eq do incremento/decremento com base no douglas Wang. 2. Explicar o quantumwalk como incremento/decremento controlado. 3. Colocar histogramas com 1,2 e 3 steps.
\subfile{Chapters/Chapter4/coinedQWQiskit}
\section{Continuous}
\subfile{Chapters/Chapter4/contQWQiskit}
\section{Staggered}
\section{Search Problems with Qiskit}
%TODO: Decidir se o grover fica aqui ou uma a parte.
\subfile{Chapters/Chapter4/searchProblemsQiskit}

% CHAPTER - Conclusion/Future Work -------------- 
\chapter{Discussions and Conclusion}
%discutir os resultados encontrados, o future work, o que falta fazer etc.
\section{Conclusions}
\section{Prospect for future work}

\bookmarksetup{startatroot} % Ends last part.
\addtocontents{toc}{\bigskip} % Making the table of contents look good.
%\cleardoublepage

%- Bibliography (needs bibtex) -%



\bibliography{bibliography}

% Index of terms (needs  makeindex) -------------
%\printindex


% APPENDIX --------------------------------------
\umappendix{Appendix}

% Add appendix chapters
\chapter{Support material}
%	\umbackcover{
%	NB: place here information about funding, FCT project, etc in which the work is %framed. Leave empty otherwise.
%	}


\end{document}
