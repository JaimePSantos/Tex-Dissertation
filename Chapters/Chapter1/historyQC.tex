\documentclass[../../dissertation.tex]{subfiles}
\begin{document}

The modern understanding of computer science was firstly announced by
\cite{turing1936} where he developed the abstract concept of what is now called
a \textit{Turing machine}. These machines are the mathematical foundation of
programmable computers, and Turing showed that there is a \textit{Universal
Turing Machine} that can be used to simulate any other Turing Machine. This
means that if an algorithm can be executed in any piece of hardware, then there
is a Universal Turing Machine that can acompish the same task. This is known as
the \textit{Church-Turing thesis}, which connects the concept of what classes
of algorithms can be run in some physical device with the mathematical
framework of a Universal Turing Machine.\par The paper published by Turing set
in motion a series of events which led to the rapid advancement of eletronic
computers and computer science. One of the earliest theoretical models
developed by John von Neumann (work later published in \cite{Neumann93} ),
presented how to assemble all the necessary parts to create a computer with all
the capabilities of a Universal Turing Machine. The true explosion of
innovation in this field came after the invention of the transistor in 1947 by
John Bardeen and Walter Brattain. The creation of the transistor led to an
unpredecent growth quantified by  \cite{moore1965} where he created
\textit{Moore's law}, stating that computer power will double with constant
cost approximately every two years. Moore's law has roughly held true
throughout the decades, by the ever increasing miniaturization of the
transistor technology. However, conventional fabrication methods run into a
problem of scale, as quantum effects begin to interfere more as the size of the
devices becomes smaller.\par

\cite{feynman1959} recognized such a miniturization was the way forward, and
even predicted the problems quantum effects presented to a classical computer.
With an amazing stroke of insight, Feynmann imagined that these effects could
be exploited given the right computational paradigm. Quantum computing begins
to take form in later work developed by \cite{benioff1980}, where the earliest
quantum mechanical model of a computer was described. In this paper, Benioff
showed that a computer working under the laws of quantum mechanics could be
used to express a Shrodinger equation description of Turing machines. Shortly
after, \cite{feynman1982} pointed out that simulating quantum systems on
classical computers is inneficient, and suggested using quantum computers for
this purpose. Additional work in the following decade further explored this
idea and showed that there are systems that quantum computers can simulate,
which have no known efficient simulation on a classical computer, and even
today this continues to be one of the most promising fields in quantum
computing.\par Driven by the work of Turing, \cite{deutsch1985} questioned if a
stronger version of the Church-Turing thesis could be derived from the laws of
physics. The strong Church-Turing thesis states that any algorithmic process
can be simulated efficiently using a probabilistic Turing machine, and Deutsh
was set to define some device that could efficiently simulate an arbitrary
physical system. Whether Deutsh's formulation of a Universal Quantum Computer
is sufficient for this function is still an open question, but what he
accomplished was a challenge to the strong Church-Turing thesis by suggesting
that there are tasks a quantum computer can accomplish efficiently that a
probabilistic Turing machine cannot. \cite{deutsch1992} present an example of a
quantum algorithm that is exponentially faster than a classical counterpart,
the \textit{Deusth-Josza algorithm } that determines if a function is constant
or balanced. Even though of little practical use, this is one of the first
examples of possible advantages a quantum computer may have over a classical
one.\par

Even though the Deutsh-Josza algorithm might not have real world applications,
it led to further research on finding other such types of algorithms.
\cite{shor1994} showed that the problem of finding prime factors of an integer
and the \textit{discrete logarithm} problem can be effieciently solved by a
quantum computer by an exponential factor. This brought a lot of interest to
quantum computing, since both of these problems have real world applications
and no efficient classical solution was/is known. Furthermore, most modern
popular algorithms used for cryptography rely on the fact that the integer
factorization or discrete logarithm problems are not effiently solvedi. Since
this is no longer the case, a new field has emerged called \textit{Post-quantum
cryptography}, whose purpose is to find suitable classical algorithms for
cryptography that are not efficiently solved by quantum computing.\par

A more modest, but very relevant advantage was presented by \cite{grover1996}
where he presented a quantum algorithm that promised to speed up unstructured
database searches quadratically. Even though it's not an exponential
improvement like Shor's algorithm, search-based algorithms are useful in many
contexts, so even a "small" quadratic gain generated a lot of interest.\par

Contemporary to computer science, information theory is another field very
relevant to this topic. \cite{shannon48} revolutionized how communication and
information are understood. In his paper, Shannon was interested in defining
what resources are requiried to send information over a communication channel
and how to reliably send that information mitigating the effects of noise. This
led to the creation of the two fundamental theorems of information theory.
Firstly, Shannon's \textit{noiseless channel coding theorem} specifies what
resources are needed to store information sent from a source. Secondly, the
\textit{noisy channel coding theorem}, specifies how much information can be
sent through a channel subject to noise. Even though Shannon's second theorem
does not define any specific methodology to reduce noise, it sets an upper
limit on how much noise can be mitigated through said methodology. These are
known as \textit{error-correcting codes} and research has developed better and
better codes that get closer and closer to Shannon's limit, and they are used
wherever there is need to store or transmit information.\par 

Similar progress was made in quantum information theory. \cite{schumacher95}
developed a quantum version of Shannon's noiseless coding theorem, where he
defined a \textit{quantum bit} as a physical resource. There is no analogue for
the second Shannon theorem, but that didn't stop the development of quantum
error-correcting theory. For example, \cite{shorcalder96} and \cite{steane96}
proposed an important class of quantum error-correcting codes known as CSS.\par

%TODO: Sera relevante?
Error-correcting was designed to protect quantum states, but another discovery
by \cite{wisnerbennet92} showed another interesting aspect about quantum
information when transmitting classical information through a quantum channel.
They explained how to send two classical bits of information using only one
qubit, in a phenomenon known as \textit{superdense coding}.\par Another
interesting application of quantum information is in the field of cryptography.
\cite{wisner60} showed how quantum mechanics could be used to make sure that a
information sent could not be interfered with without destroying it. Building
on this work, \cite{bennetbassard84} proposed a quantum key distribution
protocol between sender and receiver that could not spied upon without notice.
Many other protocols have since been proposed and experimental prototypes
developed.\par

Finally, another interesting field within quantum computation is based on the
concept of \textit{distributed quantum computation}. Quantum clusters show
promise since they require exponentially less communication to solve certain
problems, such as modeling quantum systems, but are stil in their infancy due
to technical restrictions. There has been increasing international interest in
taking advantage of these systems to build a \textit{quantum internet} which
promises better and safer transmission of information, but there are still many
technological improvements to be made before this becomes a mainstream reality. 

\end{document}
