\documentclass[../../dissertation.tex]{subfiles}
\begin{document}

The strong Church-Turing thesis, that states that any algorithmic process can be simulated efficiently using a Turing Machine, was challenged by \cite{solvaystrassen77} where they presented what is known as the \textit{Solovay-Strassen primality test}. They showed that it is possible to test whether a integer is prime or composite using a randomized algorithm. The implication is that, because of the randomness, the Solvay-Strassen primality test does not determine with certainty whether a integer is prime or composite, rather it computes that a number is \textit{probably} prime or else \textit{certainly} composite. This is of significance since no deterministic test for primality was known at the time (nor it is now), meaning that this was an example of a class of problems that could not be efficiently solved by a conventional deterministic Turing Machine.\par
%TODO: Talvez exemplos diferentes?
%TODO: Quicksort faz sentido quando tamos a falar do monte carlo?
This led to a modification of the Church-Turing thesis, now stating that any algorithm can be simulated efficiently using a \textit{probabilistic} Turing machine. The discovery of more instances of such algorithms followed,  \cite{motwani1995} and \cite{papadimitrious1994} show several problems that can be solved based on randomized algorithms, for example, the \textit{Quicksort} algorthim, developed by \cite{hoare61}, has a high probability of finishing in $O(n \log{n})$. In contrast to many deterministic algorithms that require $O(n^2)$ time. They also show algorithms that take advantage of \textit{Markov chains} and the \textit{Monte Carlo method}. The volume of a convex body, proposed by \cite{dyer1991}, can be estimated by a randomized algorithm in polynomial time; the permanent of a nonnegative entry matrix can also be approximately calculated in probabilistic polynomial time as was shown by \cite{jerrum2001} and the \textit{k-SAT} and satisfiability with restrictions problem by \cite{schoning1999}. \par
Random walks, as the name suggests, belong to this class of algorithms. \cite{kpearson1905} coined the term random walk, and they can be described as path consisting of a succession of steps determined by a stochastic process, over a mathematical space. This is a useful consctruct since random walks are used to explain the behaviour of systems across many fields, from the Brownian movement of particles moving through a gas, to the price of a fluctuating stock, as was shown by \cite{cootner67}.  
\end{document}
