\documentclass[../../dissertation.tex]{subfiles}
\begin{document}

%TODO: Como ainda nao fiz o chap2 este paragrafo esta incompleto.
The motivation behind the contents in this dissertation is to create an
expanded overview on  the topic of quantum random walks. To better
contextualize the rest of this thesis, chapter \ref{chap:QuantumCom} presents
the basic concepts and mathematical tools needed for the study of quantum
walks. \par

Chapter \ref{chap:QuantumWalks} consists of the study of three major quantum
random walk models, more specifically the discrete-time coined quantum walk,
the continuous-time quantum walk and the staggered quantum walk. For each of
these models, this work presents the theoretical framework as well as Python
implementations. In these simulations, the dynamics of the walks are analised
by changing various parameters, plotting the resulting probability
distributions and seeing how these parameters alter the shape, propagation and
other features of the quantum random walks.\par

Chapter \ref{chap:searchingProblems} follows this approach, but now the
structure where these walks take place is a complete graph and the goal is to
find a marked element. For this purpose, the Grover algorim is used to
introduce the notion of quantum searching problems, and in this section  


\end{document}

