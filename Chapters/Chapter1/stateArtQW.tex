\documentclass[../../dissertation.tex]{subfiles}
\begin{document}
%Simuladores de caminhadas quanticas 
%QUANTUM WALKS AND THEIR ALGORITHMIC APPLICATIONS (TODO: maybe?)
%% > Discretas
%% >> The QWalk Simulator of Quantum Walks (marquezino2008)
%% >> GPGPU Based Simulations for One and Two Dimensional Quantum Walks (sawerwain2010)
%% >> qwViz: Visualisation of quantum walks on graphs (berry2011) 
%% > Continuas
%% >> pyCTQW : A continuous-time quantum walk simulator on distributed memory computers (izaac2014) 
%% >> QSWalk: A Mathematica package for quantum stochastic walks on arbitrary graphs (falloon2017)
%% >> QSWalk.jl: Julia package for quantum stochastic walks analysis (falloon2017) 


% Circuitos quantum walks
%% Muitas destas refs vem do loke2016a.
%% > Discrete
%% >> DouglasWang2009
%% >> S. P. Jordan, P. Wocjan, Efficient quantum circuits for arbitrary sparse unitaries 2009 (DUVIDOSO) (jordan2009)
%% >> T. Loke, J. B. Wang, Efficient circuit implementation of quantum walks on non-degree-regular graphs 2012 (loke2012)
%% >> Asif  Shakeel.Efficient  and  scalable  quantum  walk  algorithms  via the  quantum  Fourier  transform. (shakeel2020) (TODO: isto tambem tem implementacao!!)
%% > Continuous
%% >> G. Ahokas, R. Cleve, B. C. Sanders, Efficient Quantum Algorithms for Simulating Sparse Hamiltonians (berry2006) (TODO: DUVIDOSO)
%% >> A. M. Childs, R. Kothari, Limitations on the simulation of non-sparse Hamiltonians (childs2010) (TODO: DUVIDOSO)
%% >> X. Qiang, T. Loke, A. Montanaro, K. Aungskunsiri, X. Zhou, J. L. OBrien, J. B. Wang, J. C. F. Matthews, Efficient quantum walk on a quantum processor (qiang2016)
%% >> loke2016b
%% > Szegedy
%% >> C.-F. Chiang, D. Nagaj, P. Wocjan, Efficient Circuits for Quantum Walks (chiang2009)
%% >> loke2016a

%% >> Physical Implementation of Quantum Walks. (todo: Ver onde isto se enquadra melhor)

% Implementacao (mencionar searching)
%% >> balu2017
%% >> georg2019
%% >> acasiette2020
%% >> figgatt2017 (TODO: Perceber como introduzir o grover aqui)

%% > TODO: Decidir se ponho experiencias em laboratorio
%% > R. Matjeschk, Ch. Schneider, M. Enderlein, T. Huber, H. Schmitz, J. Glueckert, and
%% > T. Schaetz. Experimental simulation and limitations of quantum walks with trapped ions.
%% > E. Flurin, V. V. Ramasesh, S. Hacohen-Gourgy, L. S. Martin, N. Y. Yao, and I. Siddiqi. Observing topological invariants using quantum walks in superconducting circuits.
%% > S. Dadras, A. Gresch, C. Groiseau, S. Wimberger, and G. S. Summy. Experimental realization of a momentum-space quantum walk.
\end{document}
