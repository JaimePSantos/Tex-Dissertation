\documentclass[../../dissertation.tex]{subfiles}
\begin{document}

One of the earliest works on a more computational approach to quantum random
walks was by \cite{marquezino2008}, where they created a general simulator for
discrete-time quantum walks on one- and two-dimensional lattices. They argued
that this framework allowed reasearchers to focus more on the mathematical
aspect of quantum walks, instead of the specific numerical implementations.
Further work on these lattices was later presented by \cite{sawerwain2010},
where they studied the simulation of quantum walks by taking advantage of the
GPU and CUDA techonlogy. Another interesting program for simulating discre-time
quantum walks came with the work of \cite{berry2011}. This package allows for
direct simulation of these walks, and visualisation of the time-evolution on
arbitrary undirected graphs. It also allowed for plotting of continuous-time
quantum walks, provided the data was provided externally. 
%TODO: Simulator Continuous : Sera que faco um paragrafo aqui?
There are, however, direct simulators of continuous-time quantum walks, the
earliest being by \cite{izaac2015}. Their distributed memory software claims to
be able to perform efficient simulation of multi-particle continuous-time
quantum walk based systems, on \textit{High Performance Computing} platforms.
\cite{falloon2017a} provide a \textit{Mathematica} package that implements a
simulator of \textit{Quantum Stochastic Walks}, which are a generalization of
the continuous-time model. These walks incorporate both coherent and incoherent
dynamics, which means that quantum stochastic walks can be instantiated as both
quantum walks and classical random walks. What this paper then provides is a
way of implementing quantum walks on directed graphs, opening the door to
applications ranging from the capture of energy by photosyntethic protein
complexes, to page ranking algorithms used by search engines. This package was
ported to and expanded in the \textit{Julia} programming language by
\cite{glos2018}.\par
%Circuit Discrete TODO: Decidir se meto loke2012 aqui.
In order to truly harness the power of quantum computing, however, one must be
able to perform these algorithms on quantum hardware. For this purpous, various
implementations of quantum walks on quantum circuits have been purposed.
\cite{douglaswang07} pioneered this approach by developing
efficient quantum circuits for discrete-time quantum walks on highly symmetric
graphs, whose resources scale logarithmically with the size of the state space.
\cite{shakeel2020} presented a new approach for building circuits for the
discrete model, reducing resource requirement by using the quantum Fourier
transform. 
%TODO: Circuit continuous: paragrafo?
For the continuous-time quantum walk, work by \cite{qiang2016} presents
efficient quantum circuits for the circulant graph class, and also an
experimental implementation on a photonic quantum processor. In the same year,
\cite{loke2017b} showed how to build continous-time quantum walk circuits for
composite graphs, namely commuting graphs and Cartesian product of graphs.
%TODO: Circuit szegedy: paragrafo?
Considering the Szegedy quantum walk, \cite{chiang2009} proposed an efficient
method of creating quantum circuits for this model.  In their work, they showed
how to derive a quantum version of the arbitrary sparse classical random walk
by approximating a \textit{quantum update rule} with circuit complexity scaling
linearly with the degree of sparseness of the structure. \cite{loke2017a}
developed this method by showing that an efficient circuit for the Szegedy
quantum walk can be constructed even if the structures are not sparse, given
they posses translational symmetry in the columns of the transitional matrix.
More specifically, they identified that the class of cyclic and bipartite
graphs are compatible with this approach.  Another interesting result in this
paper was the creation of circuits that implement a quantum analogue of
Google's \textit{Page Rank} algorithm, in terms of Szegedy walks.

% Simuladores de caminhadas quanticas 
%QUANTUM WALKS AND THEIR ALGORITHMIC APPLICATIONS (TODO: maybe?)
%% > Discretas
%% >> The QWalk Simulator of Quantum Walks (marquezino2008)
%% >> GPGPU Based Simulations for One and Two Dimensional Quantum Walks (sawerwain2010)
%% >> qwViz: Visualisation of quantum walks on graphs (berry2011) 
%% > Continuas
%% >> pyCTQW : A continuous-time quantum walk simulator on distributed memory computers (izaac2014) 
%% >> QSWalk: A Mathematica package for quantum stochastic walks on arbitrary graphs (falloon2017)
%% >> QSWalk.jl: Julia package for quantum stochastic walks analysis (falloon2017) 
%%
% Circuitos quantum walks
%% Muitas destas refs vem do loke2016a.
%% > Discrete
%% >> DouglasWang2009
%% >> S. P. Jordan, P. Wocjan, Efficient quantum circuits for arbitrary sparse unitaries 2009 (DUVIDOSO) (jordan2009) %% >> T. Loke, J. B. Wang, Efficient circuit implementation of quantum walks on non-degree-regular graphs 2012 (loke2012) %% >> Asif  Shakeel.Efficient  and  scalable  quantum  walk  algorithms  via the  quantum  Fourier  transform. (shakeel2020) (TODO: isto tambem tem implementacao!!)
%% > Continuous
%% >> G. Ahokas, R. Cleve, B. C. Sanders, Efficient Quantum Algorithms for Simulating Sparse Hamiltonians (berry2006) (TODO: DUVIDOSO)
%% >> A. M. Childs, R. Kothari, Limitations on the simulation of non-sparse Hamiltonians (childs2010) (TODO: DUVIDOSO)
%% >> X. Qiang, T. Loke, A. Montanaro, K. Aungskunsiri, X. Zhou, J. L. OBrien, J. B. Wang, J. C. F. Matthews, Efficient quantum walk on a quantum processor (qiang2016)
%% >> loke2016b
%% > Szegedy
%% >> C.-F. Chiang, D. Nagaj, P. Wocjan, Efficient Circuits for Quantum Walks (chiang2009)
%% >> loke2016a
%%
%% >> Physical Implementation of Quantum Walks. (todo: Ver onde isto se enquadra melhor)
%%
% Implementacao (mencionar searching)
%% >> balu2017
%% >> georg2019
%% >> acasiette2020
%% >> figgatt2017 (TODO: Perceber como introduzir o grover aqui)

%% > TODO: Decidir se ponho experiencias em laboratorio
%% > R. Matjeschk, Ch. Schneider, M. Enderlein, T. Huber, H. Schmitz, J. Glueckert, and
%% > T. Schaetz. Experimental simulation and limitations of quantum walks with trapped ions.
%% > E. Flurin, V. V. Ramasesh, S. Hacohen-Gourgy, L. S. Martin, N. Y. Yao, and I. Siddiqi. Observing topological invariants using quantum walks in superconducting circuits.
%% > S. Dadras, A. Gresch, C. Groiseau, S. Wimberger, and G. S. Summy. Experimental realization of a momentum-space quantum walk.

\end{document}
