\documentclass[../../dissertation.tex]{subfiles}
\begin{document}
Similarly to the continuous-time quantum walk, the staggered case aims to spread a transition probability to neighboring vertices but with discrete time steps. The notion of adjacency now comes from cliques\footnote{A clique is defined as the subset of vertices of an undirected graph such that every two distinct vertices in each clique are adjacent.}, and the initial stage of this walk consists in partitioning the graph in several different cliques. This is called tessellation, and it is defined as the division of the set of vertices into disjoint cliques. An element of a tessellation $\mathscr{T}$ is called a polygon, and it's only valid if if all of its vertices belong to the clique in $\mathscr{T}$. The set of polygons of each tessellation must cover all vertices of the graph, and the set of tessellations ($\mathscr{T}_{1}$,$\mathscr{T}_{2}$,...,$\mathscr{T}_{k}$) must cover all edges.\par
These definitions allow the construction of operators $H_1$,$H_2$,...,$H_k$ that will be used to propagate the probability amplitude locally, in each polygon. The state associated to each polygon is
\begin{equation}
	\ket{u_{j}^{k}} = \frac{1}{\sqrt{\mathopen|\alpha_{j}^{k}}\mathclose|}\sum_{l\in\alpha_{j}^{k}}\ket{l}
\end{equation}
where $\alpha_{j}^{k}$ is the $j-th$ polygon in the $k-th$ tessellation.\par
The unitary and Hermitian operator $H_k$, associated to each tessellation is defined in \cite{portugal2017b} as
\begin{equation}
	H_k = 2\sum_{j=1}^{p}\ket{u_{j}^{k}}\bra{u_{j}^{k}} - I
	\label{eq:StagHamil}
\end{equation}\par
Solving the time-independent Schrodinger equation for this Hamiltonian gives the evolution operator
\begin{equation}
	U = e^{i\theta_{k}H_{k}}...e^{i\theta_{2}H_{2}}e^{i\theta_{1}H_{1}}
	\label{eq:stagWalkUnmodOp}
\end{equation}
where
\begin{equation}
	e^{i\theta_{k}H_{k}} = \cos{(\theta_k)}I + i\sin{(\theta_k)}H_k
\end{equation}
since $H_k^2 = I$.\par
%TODO:\textcolor{red}{trocar and por since. Posso referir livro do nielsen}\par
The simplest use case of this quantum walk model is the one-dimensional lattice, where the minimum tessalations are
\begin{equation}
	\mathscr{T}_{\alpha}= \{\{2x,2x+1\}\colon x \in \mathbb{Z}\}
\end{equation}
\begin{equation}
	\mathscr{T}_{\beta}= \{\{2x+1,2x+2\}\colon x \in \mathbb{Z}\}
\end{equation}
Each element of the tessellation has a corresponding state, and the uniform superposition of these states is
\begin{equation}
	\ket{\alpha_x} = \frac{\ket{2x} + \ket{2x+1}}{\sqrt{2}}
\end{equation}
\begin{equation}
	\ket{\beta_x} = \frac{\ket{2x+1}+\ket{2x+2}}{\sqrt{2}}
\end{equation}\par
One can now define Hamiltonians $H_\alpha$ and $H_\beta$ as 
\begin{equation}
	H_\alpha = 2\sum_{x=-\infty}^{+\infty}\ket{\alpha_{x}}\bra{\alpha_x} - I
	\label{eq:stagSimulHalpha}
\end{equation}
\begin{equation}
	H_\beta = 2\sum_{x=-\infty}^{+\infty}\ket{\beta_{x}}\bra{\beta_x} - I
	\label{eq:stagSimulHbeta}
\end{equation}\par
The Hamiltonian evolution operator reduces to
\begin{equation}
	U = e^{i\theta H_\beta}e^{i\theta H_\alpha}
	\label{eq:stagSimulUniOp}
\end{equation}
and applying it to an initial condition $\ket{\Psi(0)}$ results in the time evolution operator
\begin{equation}
	U\ket{\Psi(t)} = U^t\ket{\Psi(0)}
\end{equation}\par
Having defined the time evolution operator, the walk is ready to be coded with a certain initial condition and $\theta$ value, to better understand how the probability distribution spreads through time. 
%TODO:\textcolor{red}{você pode acrescentar que o $\theta$ terá um papel similar ao $\gamma$ no controle do desvio-padrão}

For the first case study, the initial condition will be a uniform superposition of states $\ket{0}$ and $\ket{1}$ and the $\theta$ value will be varied in order to understand how this parameter impacts the walk

\begin{figure}[!h]
	\centering
	\includegraphics[scale=0.40]{img/StagQuantumWalk/stagqwMultiple.png}
	\caption{Probability distribution for the staggered quantum walk on a line after 50 steps, with initial condition $\ket{\Psi(0)}=\frac{\ket{0}+\ket{1}}{\sqrt{2}}$, for multiple angles.} 
	\label{fig:stagQWSimulMultTheta}
\end{figure}

The overall structure of the probability distribution remains the same, the difference is that the walker is more likely to be found further away from the origin as the angle increases.\par
Another interesting case study is to see how the initial condition affects the dynamics of the system

\begin{figure}[!h]
	\minipage{0.55\textwidth}
	\includegraphics[width=\linewidth]{img/StagQuantumWalk/stagqwSingle0.png}
	\caption{$\ket{\Psi(0)}=\ket{0}$}\label{fig:fig6}
	\endminipage\hfill
	\minipage{0.55\textwidth}
	\includegraphics[width=\linewidth]{img/StagQuantumWalk/stagqwSingle1.png}
	\caption{$\ket{\Psi(0)}=\ket{1}$}\label{fig:fig7}
	\endminipage\hfill
\end{figure}

Similarly to the coined case, each initial condition results in asymmetric probability distributions, $\ket{\Psi(0)}=\ket{0}$ leads to a peak  in the left-hand side while condition $\ket{\Psi(0)}=\ket{1}$ results in a peak in the right-hand side. As was shown in \ref{fig:fig5}, the uniform superposition of both these conditions results in a symmetric probability distribution.
%TODO: \textcolor{red}{acho que você pode explicar um pouco mais o papel de $\theta$ através dos gráficos e podemos pensar se fazemos gráficos do desvio-padrão para este e o contínuo}


\end{document}
