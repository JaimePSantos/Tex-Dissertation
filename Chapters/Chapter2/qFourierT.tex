\documentclass[../../dissertation.tex]{subfiles}
\begin{document}
As was seen in section \ref{sec:chapGrover}, quantum computers can perform certain tasks more efficiently than classical computers. Another such example is the problem of finding the prime factorization of an $n$-bit integer, which the most efficient solution to date, proposed by \cite{Pollard93}, requires $e^{O(n^{\frac{1}{3}}\log^{\frac{2}{3}}n)}$ operations. In contrast, a classical algorithm proposed by   accomplishes the same task in $O((\log n)^2 (\log \log n) (\log \log \log n))$ operations , which is an exponential gain due to the efficiency of the quantum Fourier transform.\par
The quantum Fourier transform is an implementation of the discrete Fourier transform over amplitudes of quantum states. It offers no speed ups when used in computing Fourier transforms of classical data, since the amplitudes cannot be accessed directly by measurement. Moreover, there is no known generalized efficient way of preparing the initial state to be Fourier Transform. What this means is that the uses of the QFT are not in the straightforward way of calculating discrete Fourier transforms, but in the form of algorithms, such as \textit{phase estimation}, that take advantage of it's properties. This transform can be described as the following operation over an orthonormal basis $\ket{0}, \ket{1},\cdots,\ket{N-1}$,
\begin{equation}
	QFT(\ket{j}) = \frac{1}{\sqrt{N}} \sum_{k=0}^{N-1} e^{\frac{2\pi i j k}{N}} {\ket{k}}.
\end{equation}
%TODO: Melhorar isto?
This can be rewritten as a product
\begin{align*}
	QFT(\ket{j_1,...j_n}) & =\\&\frac{(\ket{0} + e^{2 \pi i 0.j_n} \ket{1}) \cdots  (\ket{0} + e^{2 \pi i 0.j_{n-1}j_n} \ket{1})}{2^{\frac{N}{2}}},
\end{align*}
which is a very useful representation because it makes constructing an efficient circuit much more simple, as well as proving that the quantum Fourier transform is unitary. However, the circuit implementation of the QFT requires exponentially smaller phase-shift gates as the number of qubits increases.
\end{document}
