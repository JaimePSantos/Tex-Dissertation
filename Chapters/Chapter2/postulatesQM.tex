\documentclass[../../dissertation.tex]{subfiles}
\newtheorem{post}{Postulate}
\begin{document}

%TODO:Referencia sem ser do sakurai? 
Quantum mechanics, firstly discovered in the decade of 1920, is a mathematical
framework for developing physical theories. In this section, the goal is to
provide the basic postulates of quantum mechanics, which are formalized through
Dirac notation. Further reading on this notation includes the work of
\cite{sakurai1994} or *liboff*, and for an extensive review of quantum computation the
book by \cite{nielsen2011}.\par  
The first postulate defines where the processes of quantum mechanics take
place. The \textit{state} of a system describes it's physical characteristics,
so some rules are required for these mathematical objects to have a connection
to the real world.
\begin{post}[State Space] 
	Any isolated physical system has an associated Hilbert Space,
	$\mathcal{H}$, known as the state space. The state of the system is
	wholly described by it's state vector $\ket{\psi} \in \mathcal{H}$. The
	physical system's degrees of freedom dictate the dimension of
	$\mathcal{H}$\par
\end{post}
Note that this postulate does not tell us the Hilbert space of any given
physical system, nor does it tell us it's state vector. It is generally hard to
define the Hilbert space of an arbitrary system, which makes the work physicist
have done developing certain theories, like quantum eletrodynamics, even more
remarkable.\par 
Considering the computational basis $\{\ket{0},\ket{1}\}$, the simplest quantum
system, the \textit{qubit}, can be defined as 
\begin{equation}
	\ket{\psi} = \alpha \ket{0} + \beta \ket{1},
\end{equation}
where $\alpha,\beta$ are complex numbers and
\begin{equation}
	\ket{0} = \begin{pmatrix} 1 \\0 \end{pmatrix}  \; \; \text{and}  \;  \;  \ket{1} = \begin{pmatrix} 0 \\1 \end{pmatrix}.
\end{equation}
Because Hilbert spaces are also vector spaces, linear combinations of these
states are also allowed 
\begin{equation}
	\ket{\psi} = \sum_i \alpha_i \ket{\varphi_i}.
\end{equation}
Since $\ket{\psi}$ is required to be a unit vector, $\ket{\psi}\bra{\psi}=1$ or, equivalently,
\begin{equation}
	\sum_j | \alpha_j |^2 = 1 .
\end{equation}
These combinations are the main difference between a classical bit and a qubit,
known as \textit{superpositions}, in which the states are not in a definite
value before measurement. This is a well studied quantum phenomenon that leads
to constructive and destructive interference between states, which is an aspect
of quantum computation many algorithms exploit.\par

The second postulate aims to describe how a quantum system evolves with time,
and it can be formulated in the following way. 
\begin{post}[Evolution]
	\label{post:evolution}
	The time evolution of a closed quantum system is described by a unitary
	operator. Considering an initial condition $\ket{\psi_0}$, then for any
	time evolution of a closed quantum system, a unitary operator $U$ exists
	such that $\ket{\psi_f}=U\ket{\psi_0}$.\par
\end{post}
Just as the first postulate does not specify a Hilbert space, the evolution
postulate does not state which unitary operators $U$ describe an arbitrary
physical system. What it does state is that the evolution of a closed quantum
system follows those rules. More specifically, the second postulate describes
the dynamics of such systems. In the case of a qubit, any U can be performed in
a realistic system as long as $U U^\dag = U^\dag U = I$, which is another way
of saying that $U$ is unitary.\par  
%TODO: Esta prox frase esta fraca.
There are a few unitary operators very relevant to quantum computation, known
as \textit{Pauli matrices}, that are usually composed to create more general
matrices. These are defined as
\begin{equation}
	I = \begin{pmatrix} 
		1 && 0\\
		0 && 1
	    \end{pmatrix}, \; \;
	X = \begin{pmatrix} 
		0 && 1\\
		1 && 0
	    \end{pmatrix}, \; \;
	Y = \begin{pmatrix} 
		0 && -i\\
		i && 0
	    \end{pmatrix} \;\; and \; \;
	Z = \begin{pmatrix} 
		1 && 0\\
		0 && -1
	    \end{pmatrix}.
\end{equation}\par
Postulate \ref{post:evolution} tells us the relationship between the states of
the system at two different times. An improved version of this postulate takes
time as a continuous variable, stating that the temporal evolution of a closed
quantum system can be described by the Schrodinger equation 
\begin{equation}
	i \hbar \frac{d\ket{\psi}}{dt} = H \ket{\psi},
\end{equation}
%TODO: Decidir se devo falar sobre os valores proprios de energia e assim.
where $\hbar$ is the \textit{Planck's constant} and $H$ is a hermitian operator
known as the \textit{Hamiltonian} of the system. In principle, the Hamiltonian
can be used to describe a system in it's entirety, however figuring out the
Hamiltonian is generally a hard task.\par

Thus far only single quantum systems have been considered. The next postulate
describes how one can create composite quantum systems made of smaller distinct
systems.
\begin{post}[Composite Systems]
	The state space of a system composed of smaller sub-systems can be
	described by the tensor product of the individual state spaces
	$\mathcal{H}_1\otimes\mathcal{H}_2$. Moreover, if the first system's
	state is $\ket{\psi_1}$ and the second is $\ket{\psi_2}$ then the state
	of the composite system is $\ket{\psi_1} \otimes \ket{\psi_2}$.  
\end{post}\par
%TODO: O prox paragrafo nao esta muito bom.
The tensor product is used because of the nature of superposition in quantum
mechanics. A system composed of subsystems $\{\ket{\psi},\ket{\varphi}\}$ can
be denoted as $\ket{psi}\otimes\ket{\varphi}$. Applying the superposition principle,
that tells us that any complex linear combination of states belonging to the
system should also be allowed, the tensor product naturally follows.\par
The tensor notation can be written in a more compact way. Considering a system
described by $n$ component states \begin{equation}
	\ket{\psi} = \ket{\psi_0} \otimes \ket{\psi_1} \otimes \cdots \otimes \ket{\psi_{n-1}},
\end{equation}
it can be rewritten as 
\begin{equation}
	\ket{\psi} = \ket{\psi_0}\ket{\psi_1} \cdots \ket{\psi_{n-1}},
\end{equation}
and even further compacted to
\begin{equation}
	\ket{\psi} = \ket{\psi_0 \psi_1 \cdots \psi_{n-1}}.
\end{equation}\par
However, not all composite systems can be described as the tensor product of
the component states. For example, state 
\begin{equation}
	\label{eq:post3BellState}
	\ket{\psi} = \frac{\ket{00} +\ket{11}}{\sqrt{2}}, 
\end{equation}
cannot be broken down further. These subsytems are known to be
\textit{entangled}, and the state in equation \ref{eq:post3BellState} is called
a \textit{Bell} state. Bell states describe the set of $2$ qubit states that
present maximum entangled, and are used in many quantum aplications like
\textit{quantum teleportation} and \textit{superdense coding}.

\begin{post}[Measurement]
	Bla.
\end{post}
\end{document}
