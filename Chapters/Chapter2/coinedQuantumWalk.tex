\documentclass[../../dissertation.tex]{subfiles}
\begin{document}
% 	    \subsection{[TEMP Artigos relevantes] }
% 	        MAIS MOEDAS.
% 	        aharonov1993 - Pasta introqw/Coin \\
% 	        \textbf{aharonov2001 - Pasta introqw/Coin} - definiçao da QW  ;shift operator mod n \\
% 	        ambainis2001 - Pasta introqw/Coin \\
% 	        carteret2005- Implementacao - Pasta introqw/Coin \\
% 	        inui2003 -  - Pasta introqw/Coin \\
% 	        kendon2006 - Introducao historia CQ - Pasta introqw/Coin \\
% 	        shenvi2002 - Relacao com grover pg. 10\\
% 	        Resto da pasta.

% 	    \subsection{Texto}



%TODO: IMPORTANTE! Substituir as captions das figuras.
%TODO: IMPORTANTE! Falar do espaco de hilbert. Decidir se aqui ou noutro capitulo. 
%TODO: Este primeiro paragrafo nao esta muito bom.
In the quantum case, the walker is a quantum system whose position on a discretely numbered line, is described by a vector $\ket{x}$ in Hilbert Space. The next position of the system will depend, in part, of a unitary operator, which can be viewed as a quantum coin.
%TODO: \textcolor{red}{O operador unitário não é somente da moeda, precisa falar do espaço do caminhante também}
The analogy is, if the coin is tossed and rolls "heads", for example, the system transitions to position $\ket{x+1}$, otherwise it advances to $\ket{x-1}$. From a physical perspective, this coin can be the spin of an electron or the chirality of a particle, for example, and the outcome of measuring these properties decides whether the walker moves left or right. The coin is a unitary operator defined as
%TODO: Decidir se ponho as letras ou explico de outra maneira.
\begin{equation}
	\begin{cases}
		C\ket{0}\ket{x} = a\ket{0}\ket{x} + b\ket{1}\ket{x}\\
		C\ket{1}\ket{x} = c\ket{0}\ket{x} + d\ket{1}\ket{x},
	\end{cases}
\end{equation}
where $a$, $b$, $c$ and $d$ are the amplitudes associated with each outcome of the coin toss. One of the most commonly used coins is the unbiased coin, also known as Hadamard operator
\begin{equation}
	H = \begin{pmatrix} 
		a & c\\
		b & d
	    \end{pmatrix}
	    =\frac{1}{\sqrt{2}} \begin{pmatrix}
			    		1 & 1\\
					1 & -1
	   		       \end{pmatrix},
\end{equation}
which will be the one used in this example.\par 
The Hilbert space of the system is $\mathscr{H} = \mathscr{H}_{C} \otimes \mathscr{H}_{P}$, where $\mathscr{H}_{C}$ is the two-dimensional Hilbert space associated with the coin and $\mathscr{H}_P$ is the Hilbert space of the walker.\par
The transition from $\ket{x}$ to either $\ket{x+1}$ or $\ket{x-1}$ must be described by a unitary operator, the \textit{shift operator} 
%TODO: \textcolor{red}{precisa descrever a moeda também e como ela funciona no unitário e acredito que as equações estejam trocadas}
\begin{equation}
	\begin{cases}
		\mathcal{S} \ket{0}\ket{x} = \ket{0}\ket{x+1}\\
		\mathcal{S} \ket{1}\ket{x} = \ket{1}\ket{x-1},
	\end{cases}
\end{equation}
that can also be described by
\begin{equation}
	S = \ket{0}\bra{0} \otimes \sum_{x=-\infty}^{x=\infty} \ket{x+1}\bra{x} + \ket{1}\bra{1}\otimes \sum_{x=-\infty}^{x=\infty} \ket{x-1}\bra{x}.
	\label{eq:coinedShiftOperator}
\end{equation}
It follows that the operator that describes the dynamics of the quantum walk will be given by 
%TODO: \textcolor{red}{Não necessariamente é o Hadamard, poderia ser uma coin qualquer. Você pode explicar também que a H é a sem viés.}
%TODO: Perceber melhor a deducao do operador unitario e decidir se o demonstro aqui.
\begin{equation}
	U = S(C\otimes I) = S(H\otimes I). 
	\label{eq:coinedUnmarkedOperator}
\end{equation}\par
Consider a quantum system located at $\ket{x = 0}$ with coin state $\ket{0}$, for $t=0$. It's state will be described by
\begin{equation}
	\ket{\Psi(0)} = \ket{0}\ket{x=0}.
	\label{eq:coinedQWInitCond0}
\end{equation}
After $t$ steps 
\begin{equation}
	\ket{\Psi(t)}=U^{t}\ket{\Psi(0)},
\end{equation}
more explicitly
\begin{equation}
	\ket{\Psi(0)}\xrightarrow[]{\text{$U$}}\ket{\Psi(1)}\xrightarrow[]{\text{$U$}}\ket{\Psi(2)}\xrightarrow[]{\text{$U$}} (...) \xrightarrow[]{\text{$U$}}\ket{\Psi(t)}.
\end{equation}\par
In other words, the coined quantum walk algorithm consists on applying the coin operator followed by the shift operator a certain number of times. Iterating this twice, evolves the system to the following respective states
%TODO: \textcolor{red}{seria bom especificar que é na moeda, ou até mesmo dizer que o quantum walk consiste da aplicação da moeda e logo em seguida o shift}
\begin{gather}
	\ket{\Psi(1)} = \frac{\ket{0}\ket{x=-1}+\ket{1}\ket{x=1}}{\sqrt{2}}\label{eq:10}\\
	\ket{\Psi(2)} = \frac{\ket{0}\ket{x=-2}+\ket{1}\ket{x=0}+\ket{0}\ket{x=0}-\ket{1}\ket{x=2}}{2}\label{eq:11}\\
\end{gather}
If one were to measure the system after the first application of $\mathcal{U}$, it would be expected to see the walker at $x=1$ with probability $P(x) = \frac{1}{2}$, and at $x=-1$ with $P(x) = \frac{1}{2}$ aswell. Measure the system $t$ times, after each application of $\mathcal{U}$, and the result is a binomial probability distribution similar to the one in \ref{fig:MultClassicalWalk72180450}. The conclusion is that repetitive measurement of a coined quantum walk system reduces to the classical case, which means that any desired quantum behaviour is lost. \par
It is possible, however, to make use of the quantum correlations between different positions to generate constructive or destructive interference, by applying the Hadamard and shift operators successively without intermediary measurements.
The consequences of interference between states become very apparent after only 3 iterations 
\begin{equation}
	\ket{\Psi(3)} = \frac{\ket{1}\ket{x=-3}-\ket{0}\ket{x=-1}+2(\ket{0}+\ket{1})\ket{x=1}+\ket{0}\ket{x=3}}{2\sqrt{2}}\label{eq:12}.
\end{equation}
%TODO: Decidir se deixo ficar o sigma=0.54t ou se menciono simplesmente que e proporcional a t.
Even though an unbiased coin was used, this state is not symmetric around the origin and the probability distributions will not be centered in the origin. Moreover, \cite{REN1} shows that the standard deviation will be
\begin{equation}
	\sigma(t) \approx 0.54t .
\end{equation}
This means that the standard deviation for the coined quantum walk grows linearly in time, unlike the classical case which grows with $\sqrt{t}$, as was seen in equation \ref{eq:classicalWalkDeviation}. The implication is that the quantum walk displays \textit{ballistic} behaviour, as is reviewed in \cite{andraca2012}. This behaviour is usually defined in the context of a moving free particle with unit velocity in a single direction, which is expected to be found at $x=t$ after $t$ steps. The velocity of a walker in a Hadamard quantum walk is approximately half of the free particle example, which is still a quadratic improvement over the classical random walk.\par
\begin{figure}[!h]
	\centering
	\includegraphics[scale=0.40]{img/CoinedQuantumWalk/CoinedMultiple_psi0_3264128.png}
	\caption{Probability distribution for the coined quantum walk on a line, after 100 steps, with initial condition $\ket{\Psi(0)}=\ket{0}\ket{x=0}$ and the Hadamard coin.} 
	\label{fig:coinedQWDist0}
\end{figure}
This quadratic gain implies exponentially faster hitting times in certain graphs, as shown by \cite{childs2002}, meaning improvements to problems that require transversing graphs. \cite{ambainis2003} also shows advantages of the coined quantum walk model in element distinctness problems, and \cite{childs2004} show advantages in spatial search problems, which will be studied in a later chapter.\par
%TODO: \textcolor{red}{Bom, faltou dizer qual o desvio padrão da quantum walk que é $t$ e esse $\sqrt{t}$ é da clássica. Precisa justificar com alguma referência também}\par
%The quantum walk is said to be \textit{ballistic} since its standard deviation is proportional to $t$ \cite{andraca2012} meaning exponentially faster hitting times in certain graphs \cite{childs2002,fahri98} which can be advantageous in problems that involve visiting certain vertices in a graph. There are also studies that show that a quantum walk may have advantages in element distinctness \cite{ambainis2003} and spatial search \cite{childs2004} problems. 
%TODO: \textcolor{red}{talvez esse parágrafo deva subir, porque você fala de desvio padrão muito antes}
In order to study this distribution, a simulation of the coined quantum walk was coded in \textit{Python}. Figure \ref{fig:coinedQWDist0} is the result of using the Hadamard coin and the initial condition in equation \ref{eq:coinedQWInitCond0}, for varying numbers of steps. Analyzing the plot, it is noticeable that the distributions are asymmetric. The probability of finding the walker on the right-hand side is much larger than on the left, with a peak around $x \approx \frac{t}{\sqrt{2}}$. Regardless of number of steps, this peak is always present (albeit in varying positions), which is to say that the walker can always be found moving in a uniform fashion away from the origin, consistent with ballistic behaviour.\par 
%TODO: \textcolor{red}{agora penso que um grafo sobre desvio padrão seria bom para ilustrar, porque o desvio padrão foi muito comentado ao longo do texto}\par
Another interesting case study is to find if this behaviour is preserved for a symmetric distribution around the origin. For this purpose, one must first understand where the asymmetry comes from.
\begin{figure}[!h]
	\centering
	\includegraphics[scale=0.40]{img/CoinedQuantumWalk/CoinedMultiple_psi1_3264128.png}
	\caption{Probability distribution for the coined quantum walk on a line, after 100 steps, with initial condition $\ket{\Psi(0)}=-\ket{1}\ket{x=0}$ and the Hadamard coin.} 
	\label{fig:coinedQWDist1}
\end{figure}
The Hadamard operator flips the sign of state $\ket{1}$, hence more terms are cancelled when the coin state is $\ket{1}$. Since $\ket{0}$ was defined to induce movement to the right, the result is as shown in \ref{fig:coinedQWDist0}. Following this logic, it would be expected that an initial condition 
%TODO: \textcolor{red}{cuidado com a notação embaixo dos kets, e esse sinal negativo é necessário? acredito que não.} Pagina 30 do renato, perceber melhor o sinal.
\begin{equation}
	\ket{\Psi(0)} = \ket{1}\ket{x=0},
	\label{eq:coinedQWInitCond1}
\end{equation}
would result in more cancellations when the coin state is $\ket{0}$, thus the walker would be more likely found in the left-hand side of the graph. This is indeed what happens, as figure \ref{fig:coinedQWDist1} is a mirror image of figure \ref{fig:coinedQWDist0}. The walker still moves away from the origin with ballistic behaviour, but in opposite direction. The peaks behave in a similar fashion, being instead found at $x \approx -\frac{t}{\sqrt{2}}$.\par  
In order to obtain a symmetrical distribution, one must superpose the state in equation \ref{eq:coinedQWInitCond0} with the state in equation \ref{eq:coinedQWInitCond1}. However, in order to not cancel terms before the calculation of the probability distribution, one must multiply state $\ket{1}$ with the imaginary unit, $i$ 
%TODO:\textcolor{red}{o sinal pode ser justificado apenas aqui}
%TODO: Justificar que o sinal de i pode ser +-.
\begin{equation}
	\ket{\Psi(0)} = \frac{\ket{0}+i\ket{1}}{\sqrt{2}}\ket{x=0}.
	\label{eq:12}
\end{equation}
%TODO: muito parecido ao renato.
This works because the entries of the Hadamard operator are real numbers. Terms with the imaginary unit will not cancel out with terms without it, thus the walk can proceed to both left and right, as it is shown in figure \ref{fig:coinedQWDist01}. 
\begin{figure}[!ht]
	\centering
	\includegraphics[scale=0.40]{img/CoinedQuantumWalk/CoinedMultiple_psi01_3264128}
	\caption{Probability distribution for the coined quantum walk on a line, after 100 steps, with initial condition $\ket{\Psi(0)}=\frac{\ket{0}-i\ket{1}}{\sqrt{2}}\ket{x=0}$ and the Hadamard coin.} 
	\label{fig:coinedQWDist01}
\end{figure}\par
%TODO: Talvez escrever mais um bocado sobre a figura?
The probability distribution is now symmetric and it is spread over the range $[-\frac{t}{\sqrt{2}},-\frac{t}{\sqrt{2}}]$ with peaks around $x \approx \pm \frac{t}{\sqrt{2}}$. This means that if the position of the walker was measured at the end, it would be equally probable to find him either in the left side or the right side of the graph , which is not possible in a classical diffusive motion.\par
All of the previous examples are in sharp contrast with the classical random walk distribution in figure \ref{fig:MultClassicalWalk72180450}. There, maximum probability is reached at $x=0$ since there are approximately equal steps in both directions. Furthermore, the further the vertex is away from the origin, the less likely the walker is to be found there. However, in the quantum case, the walker is more likely to be found away from the origin as the number of steps increases. More specifically, the walk spreads quadratically faster than the classical counterpart.\par
%TODO: Melhorar o preview das proximas seccoes.
This is but one model of a quantum random walk. As it will be seen in further sections, there are other approaches to creating both discrete and continuous quantum walk models that do not use a coin. 


%TODO: \textcolor{red}{precisa de um fechamento para esta parte, algo que sumarize os resultados, o que veremos nas próximas partes, etc}


\end{document}
