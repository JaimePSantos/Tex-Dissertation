\documentclass[../../dissertation.tex]{subfiles}
\begin{document}

As was seen in section \ref{sec:chap3StagWalk}, the elements of each
tesselation of a discretely numbered cycle can be described by states
\begin{equation}
        \ket{\alpha_x} = \frac{\ket{2x} + \ket{2x+1}}{\sqrt{2}}
\end{equation}
\begin{equation}
        \ket{β_x} = \frac{\ket{2x+1}+\ket{2x+2}}{\sqrt{2}}.
\end{equation}
These states allow the construction of the Hamiltonians
\begin{equation}
        H_\alpha = 2\sum_{x=-\infty}^{+\infty}\ket{\alpha_{x}}\bra{\alpha_x} - I
\end{equation}
\begin{equation}
        H_\beta = 2\sum_{x=-\infty}^{+\infty}\ket{\beta_{x}}\bra{\beta_x} - I
\end{equation}
as was seen in equations \ref{eq:stagSimulHalpha} and \ref{eq:stagSimulHbeta}.
As was shown by \cite{acasiete2020}, these operators cane be rewritten in
matrix form 
\begin{equation}
	H_\alpha = I \otimes X
\end{equation}
%TODO: ESCREVER EQUACAO MATRICIAL DO HBETA
\begin{equation} 
	H_\beta = 
\end{equation} 
which are very useful representations when constructing the circuit.\par 

As was shown in equation \ref{q:stagSimulUniOp}, the unitary evolution operator
is
%TODO: DESCOBRIR SE A OS US ESTAO NA ORDEM CORRETA. 
\begin{equation}
	U = e^{i\theta H_\beta}e^{i\theta H_\alpha} = U_\beta U_\alpha, 
\end{equation} 
and knowing that
%TODO: TENTAR DESCOBRIR MELHOR MANEIRA DE REPRESENTAR ISTO. SERA QUE PRECISO DA MATRIZ TODA?  
\begin{equation} 
	R_x(\theta) = e^{\frac{-i\theta X}{2}, 
\end{equation} 
then each of the operators associated with the different tesselation
Hamiltonians can be written as
\begin{equation} 
	U_\alpha = I \otimes R_x(\theta) 
\end{equation}
%TODO: Escrever Beta
\begin{equation} 
	U_\beta = 
\end{equation}


%TODO: 
\end{document}
