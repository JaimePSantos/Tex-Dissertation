\documentclass[../../dissertation.tex]{subfiles}
\begin{document}

\subsection{Grover}
Like it was seen in section \ref{sec:chapGrover} Grover's algorithm is a tool that provides quadratic speedups to unstructured search problems. The routine consists on applying an oracle
\begin{equation}
        \mathcal{O}\ket{x} = (-1)^{f(x)}\ket{x},
\end{equation}\par
that flips the solution states followed by an amplitude amplification process by means of the diffusion operator 
\begin{equation}
        %TODO: Decidir se mantenho os H's.
        \mathcal{D} = (2\ket{\Psi_0}\bra{\Psi_0} - I) = H^{\otimes n}(2\ket{0}\bra{0} - I)H^{\otimes n},
\end{equation}
where $\ket{\Psi_0}  = \frac{1}{\sqrt{N}}\sum_{x=0}^{N-1} \ket{x}$. The unitary operator that describes the algorithm will then be
\begin{equation}
        \mathcal{U} = \mathcal{D}\mathcal{O}.
\end{equation}
As was before mentioned, this evolution will be done several times, depending on the number of qubit.Optimal probability of success finding a single solution will be reached after approximately $\sqrt{N}$ steps, and $\sqrt{\frac{N}{K}}$ for $K$ solutions.

%\begin{figure}[!h]
%	\centering
%	\includegraphics[scale=0.40]{img/Qiskit/GroverQiskit/GroverQiskitSearch_N3_M1_S012}
%	\caption{Probability distribution for the staggered quantum walk on a line after 50 steps, with initial condition $\ket{\Psi(0)}=\frac{\ket{0}+\ket{1}}{\sqrt{2}}$, for multiple angles.} 
%	\label{fig:fig5}
%\end{figure}
%
\subsection{Coined}
%
%\begin{figure}[!h]
%	\centering
%	\includegraphics[scale=0.40]{img/Qiskit/CoinedQuantumWalk/Search/CoinedQiskitSearch_N4_M1_S01234}
%	\caption{Probability distribution for the staggered quantum walk on a line after 50 steps, with initial condition $\ket{\Psi(0)}=\frac{\ket{0}+\ket{1}}{\sqrt{2}}$, for multiple angles.} 
%	\label{fig:fig5}
%\end{figure}

%TODO: Fazer com outra moeda usando u3 com valores intermedios entre pi/2 ou pi/4
\subsection{Continuous}
\subsection{Staggered}

\end{document}
