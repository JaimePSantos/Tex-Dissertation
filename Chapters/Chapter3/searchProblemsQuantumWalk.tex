\documentclass[../../dissertation.tex]{subfiles}
\begin{document}
In classical computation, a \textit{spatial search problem} focuses on finding marked points in a finite region of space. Defining this region with graphs is fairly straightforward, the vertices of the graph are the search space, and the edges define what transitions are possible through the search space. As was previously mentioned in \ref{chapGrover}, exhaustively searching through an unstructured space, by means of a classical random walk for example, would mean that in the worst case, one would have to take as many steps to find the marked points as there are vertices in the graph. Quantum computing provides an alternative to this complexity through Grover's algorithm, and applying some of his ideas to the coined quantum walk not only allows a quantum counterpart to the random walk search, but also further insight into the algorithm itself.\par
%TODO:\textcolor{red}{acho que você precisa delimitar o que fará ao longo desta seção, dizer que tratará os três modelos e a busca}
\subsection{Coined} \label{sec:chap3CoinedSearch}
Following \cite{REN1}'s definition, a good first step is to borrow the diffusion from Grover's algorithm and invert the sign of the state corresponding to the marked vertex while leaving unmarked vertices unchanged. This is done through the following operator 
%TODO:\textcolor{red}{eu trocaria a notação de $\mOathcal{F}$ por $\mathcal{O}$ e dizer que é um oráculo}
\begin{equation}
	\mathcal{O} = I - 2 \sum_{x\in M} \ket{x}\bra{x}
\end{equation}
where M is the set of marked vertices and $\mathcal{O}$ is an analogue to Grover's oracle. For one marked vertex, this oracle can be written as 
%TODO:\textcolor{red}{você pode até dizer que $M=\{0\}$}
\begin{equation}
	\mathcal{O} = I - 2 \ket{0}\bra{0}
\end{equation}
Notice that there is no loss of generality by choosing the marked vertex as $0$, since the labelling of the vertices is arbitrary.\par
The next step is to combine the evolution operator from the coined quantum walk model with the oracle
\begin{equation}
	U'= U\mathcal{O}
	\label{eq:43}
\end{equation}
Similarly to the simple coined case, the walker starts at $\ket{\Psi(0)}$ and evolves following the rules of an unitary operator $U$ followed by the sign inversion of marked vertices. The walker's state after an arbitrary number of steps will be
\begin{equation}
	\Psi(t) = (U')^t\ket{\Psi(0)}.
	\label{eq:sysStateSearch}
\end{equation}\par

For a better understanding of the search problem in the coined quantum walk model, consider a graph where all the vertices are connected and each vertex has a loop that allows transitions to itself, as shown in figure \ref{fig:undCompGraph}. 
%TODO: \textcolor{red}{não precisa dizer que são arcos também}
%                \begin{figure}[!h]
%                \centering
%                \begin{tikzpicture}[shorten >=1pt,auto,node distance=3cm,
%                    thick,main node/.style={circle,draw,font=\sffamily\Large\bfseries}]
%
%                      \node[main node] (1) {1};
%                      \node[main node] (2) [below left of=1] {2};
%                      \node[main node] (3) [below right of=2] {3};
%                      \node[main node] (4) [below right of=1] {4};
%                    
%                      \path[every node/.style={font=\sffamily\small}]
%                        (1) edge node [left] {} (4)
%                            edge [loop above] node {} (1)
%                            edge node [below] {} (3)
%                        (2) edge node [right] {} (1)
%                            edge node {} (4)
%                            edge [loop left] node {} (2)
%                        (3) edge node [right] {} (2)
%                            edge [loop below] node {} (3)
%                        (4) edge node [left] {} (3)
%                            edge [loop right] node {} (4)
%                 \end{tikzpicture}
%                  \caption{Undirected Complete Graph}
%                  \label{fig:undCompGraph}
%                \end{figure}
%                \textcolor{red}{acho que não há necessidade das figuras e dizer que é com grafo direcionado}
%                Grover's algorithm is equivalent to a coined quantum walk on a directed complete graph with loops such as the one in figure \ref{fig:dirCompGraph}.
%                \begin{figure}[!h]
%                    \centering
%                    \begin{tikzpicture}[->,>=stealth',shorten >=1pt,auto,node distance=3cm,
%                    thick,main node/.style={circle,draw,font=\sffamily\Large\bfseries}]
%                      \node[main node] (1) {1};
%                      \node[main node] (2) [below left of=1] {2};
%                      \node[main node] (3) [below right of=2] {3};
%                      \node[main node] (4) [below right of=1] {4};
%                      \path[every node/.style={font=\sffamily\small}]
%                        (1) edge node [bend left] {} (4)
%                            edge [bend right] node[left] {} (2)
%                            edge [loop above] node {} (1)
%                            edge node [arc below] {} (3)
%                        (2) edge node [bend right] {} (1)
%                            edge node {} (4)
%                            edge [loop left] node {} (2)
%                            edge [bend right] node[left] {} (3)
%                        (3) edge node [right] {} (2)
%                            edge [bend right] node[right] {} (4)
%                            edge node [bend up] {} (1)
%                        (4) edge node [left] {} (3)
%                            edge [loop right] node {} (4)
%                            edge [bend right] node[right] {} (1)
%                            edge node [bend left] {} (2);
%                    \end{tikzpicture}
%                    \caption{Directed complete graph with N=4 nodes.}
%                    \label{fig:dirCompGraph}
%                \end{figure}    
%                
The next step is to label the arcs using notation $\{(v,v'), v \geqslant 0 \land v' \leqslant N-1\}$ 
where $N$ is the total number of vertices and $(v,v')$ are the position and coin value, respectively, in the coined model. 
%sera que vale a pena desenhar um grafo com labels?
The shift operator, now called \textit{flip-flop} shift operator, is
\begin{equation}
	S\ket{v1}\ket{v2} = \ket{v2}\ket{v1}.
	\label{eq:chap3FlipFlop}
\end{equation}\par
The coin operator is defined as
\begin{equation}
	C = I_N \otimes G
\end{equation}
where 
%TODO:\textcolor{red}{prefiro s a D}
\begin{equation}
	G = 2\ket{D}\bra{D} - I
\end{equation}
is the Grover coin with $\ket{D}$ being the diagonal state of the coin space. Given both of these operators, the evolution is defined for the unmarked case similarly to \ref{coinedUnmarkedOperator}
\begin{equation}
	U = S(I \otimes G).
\end{equation}\par
Marking an element in a complete graph is done through the following oracle
\begin{equation}
	\mathcal{O'} =\mathcal{O}\otimes I = (I_N - 2\ket{0}\bra{0})\otimes I_N = I_{N^2} - 2 \sum_v \ket{0}\ket{v}\bra{0}\bra{v} ,
\end{equation}
that can be seen, in the arc notation, as an operator that marks all arcs leaving $0$.\par
Recalling \ref{eq:43}, the modified evolution operator can be written as
\begin{equation}
	U' = S(I \otimes G)\mathcal{O'} = S(I \otimes G)\mathcal{O} \otimes I = S (\mathcal{O} \otimes G),\label{eq:modifiedEvoCoined}
\end{equation}
and the state of the system will evolve according to equation \ref{eq:sysStateSearch}.\par
As was shown in \cite{REN1}, maximum probability of the marked vertex is achieved after $\frac{\pi}{2}\sqrt{N}$ steps. Figure \ref{fig:coinedSearch} is the result of coding and plotting the evolution of this probability distribution, for graphs of varying sizes. It shows that the probability is close to one at \textit{approximately} the predicted ideal steps, because of the discrete nature of the walk. The probability distributions have a stair-like shape, because transitions in this model only occur on even numbered time steps, because of how the unmodified evolution operator was constructed.

\begin{figure}[!h]
	\centering
	\includegraphics[scale=0.40]{img/CoinedQuantumWalk/Search/CoinedSearch163264.png}
	\caption{Discrete-time coined quantum walk search for a complete graph with 16, 32 and 64 nodes.}\label{fig:coinedSearch}
\end{figure}

\subsection{Staggered}
The process for defining the search problem in this model is similar to the coined quantum walk case. The oracle still inverts the sign of a certain state and amplifies it, and the system's state will still be described by equation \ref{eq:sysStateSearch}. However,instead of using a coin, the staggered model takes advantage of the notions of cliques and tessellations, as was shown in chapter \ref{stagWalk}, which means the unmodified evolution operator has to be defined for an undirected complete graph.\par
As was shown in figure \ref{fig:undCompGraph}, the vertices in a complete graph are all neighbors. This is a special case because this is the only connected graph where the tessellation cover can be done by one tessellation, since the graph is it's own clique. The minimum tessellations required to cover this structures are defined by the one clique that encompasses all $N$ nodes of the graph
\begin{equation}
	\mathscr{T}_{\alpha} = \{\{0,1,2,...,N-1\}\}.
\end{equation}\par
The associated polygon can then be described as the balanced superposition of all the nodes in the graph
\begin{equation}
	\ket{\alpha} = \frac{1}{\sqrt{N}} \sum_{v=0}^{N-1} \ket{v}.
\end{equation}\par
The Hamiltonian, as defined in \ref{eq:StagHamil}, is 
%TODO: \textcolor{red}{falta índice no somatório}
\begin{equation}
	H_\alpha = 2\sum_0^1\ket{\alpha}\bra{\alpha} - I = 2\ket{\alpha_0}\bra{\alpha_0} - I
\end{equation}\par
The unmodified evolution operator from equation \ref{eq:stagWalkUnmodOp}
\begin{equation}
	U = e^{i\theta_{k}H_{k}}...e^{i\theta_{2}H_{2}}e^{i\theta_{1}H_{1}}
\end{equation}
reduces to the single Hamiltonian case
\begin{equation}
	U = e^{i\theta H_\alpha}.
	\label{eq:stagQWSearchUnmodEvo1}
\end{equation}\par
The choice of the $\theta$ value is an important one, since maximum probability is achieved at $\theta = \frac{\pi}{2}$, as shown in figure \ref{fig:stagMultTheta}.
\begin{figure}[!h]
	\centering
	\includegraphics[scale=0.40]{img/StagQuantumWalk/Search/Theta163264.png}
	\caption{Maximum probability of the marked element as a function of the $\theta$ value plotted from $0$ to $\pi$ for number of nodes $N=64,128$ and $256$.} 
	\label{fig:stagMultTheta}
\end{figure}
Since $H_\alpha^2 = I$, equation \ref{eq:stagQWSearchUnmodEvo1} can be rewritten as
\begin{equation}
	U = e^{-i\frac{\pi}{2} H_\alpha} = \cos{\frac{\pi}{2}I} + i\sin{\frac{\pi}{2}H_\alpha} = i H_\alpha = i(2\ket{\alpha_0}\bra{\alpha_0} - I).
\end{equation}\par
Having defined the the evolution operator associated with the complete graph, the next step is to use the oracle
\begin{equation}
	\mathcal{O} = I_N - 2\ket{0}\bra{0},
\end{equation}
to create the modified evolution operator associated with the search
\begin{equation}
	U' = U\mathcal{O}.
\end{equation}\par
The walk achieves the same result as Grover's algorithm after $\frac{\pi}{4}\sqrt{N}$ steps, as shown in figure \ref{fig:StagSearch}. This plot also shows that the probabilities converge to $1$ as $N$ increases, this is because time is discretized and deviations to the ideal steps will matter less for bigger values of $N$.
\begin{figure}[!h]
	\centering
	\includegraphics[scale=0.40]{img/StagQuantumWalk/Search/163264.png}
	\caption{Staggered quantum walk search for a complete graph with 16, 32 and 64 nodes.}
	\label{fig:StagSearch}
\end{figure}

%probabilidade em pi/4 sqrt(N)
%probabilidade mais perto de 1 quanto maior o N, devido à natureza discreta da walk.

%TODO:\textcolor{red}{precisa de um fechamento também aqui}

\subsection{Continuous}
As was previously seen, the continuous-time quantum walk model is defined by an evolution operator obtained by solving Schrödinger's equation
\begin{equation}
	U(t) = e^{-iHt}.
\end{equation}
The search problem requires introducing an oracle to the Hamiltonian, that will mark an arbitrary vertex $m$ %meter que pertence a um conjunto de vertices marcados. chamar vertex onde tenho node.
\begin{equation}
	H' = -\gamma L - \ket{m}\bra{m}.
\end{equation}
Since the complete graph is a regular graph, the operator can be rewritten in terms of the adjacency matrix plus the marked element. Considering $\ket{0}$ is marked,
\begin{equation}
	U'(t) = e^{iH't} = e^{i(-\gamma L - \ket{0}\bra{0})t} = e^{i(-\gamma A + \gamma D - \ket{0}\bra{0})t} = e^{-i\gamma(A+\ket{0}\bra{0})t + i\gamma D t}.
\end{equation}
The degree matrix is again $D=dI$, which means it will commute with $A+\ket{0}\bra{0}$ and become a global phase
\begin{equation}
	U'(t) = e^{-i\gamma(A+\ket{0}\bra{0})t}e^{i\gamma D t} = \phi(t)e^{-i\gamma(A+\ket{0}\bra{0})t}.
\end{equation}\par
As was show by \cite{zalka1999}, the value of $\gamma$ is crucial for the success of the search. As $\gamma$ increases, the contribution of the marked element in the Hamiltonian decreases, and as $\gamma$ approaches $0$ the contribution of the adjacency matrix decreases. To find the optimum value, the Hamiltonian can be rewritten by adding multiples of the identity matrix to the adjacency matrix 
%TODO:\textcolor{red}{H' é uma notação ruim, confunde com derivada e acredito não ser necessário aqui}
\begin{equation}
	H' = -\gamma(A+NI) - \ket{0}\bra{0} = -\gamma N\ket{s}\bra{s} - \ket{0}\bra{0}
\end{equation}
where $\ket{s} = \frac{1}{\sqrt{N}}\sum_i \ket{i}$. Now it is obvious that, for $\gamma = \frac{1}{N}$, the Hamiltonian is $H = -\ket{s}\bra{s} - \ket{0}\bra{0}$. It's eigenstates are proportional to $\ket{s}\pm\ket{w}$ and eigenvalues are $-1 - \frac{1}{\sqrt{N}}$ and $-1 + \frac{1}{\sqrt{N}}$, respectively. This means that the evolution rotates from the state of balanced superposition to the marked vertex state in time $\frac{\pi}{\Delta E} = \frac{\pi}{2}\sqrt{N}$ which is, as was shown by \cite{zalka1999}, optimal and equivalent to Grover's algorithm. Plotting $\Delta E$ as a function of $\gamma N$, as can be seen in figure \ref{fig:gamma512}, has a minimum at $\gamma N =1$. The difference between the largest eigenvalue and second largest, plotted in the y-axis, is the smallest for a value of $\gamma N = 1 \implies \gamma =\frac{1}{N}$, which will correspond to the maximum probability for the marked vertex, in optimal steps.

\begin{figure}[h]
	\centering \includegraphics[scale=0.40]{img/ContQuantumWalk/Search/gamma512.png}
	\caption{Value of the difference between the largest eigenvalue and the second largest, plotted as a function of $\gamma N$, for $N=512$. }\label{fig:ContSearch}
	\label{fig:gamma512}
\end{figure}

Figure \ref{fig:ContSearch} shows the evolution of the probability of the marked vertex in time, which is continuous in this model. In contrast with previous models, the distributions are smooth and reach exactly one, since the walk is allowed to evolve to exactly the ideal time steps.

%figura da caminhada com passos ideais
\begin{figure}[!t]
	\centering \includegraphics[scale=0.40]{img/ContQuantumWalk/Search/163264.png}
	\caption{Continuous quantum walk search for a complete graph with 16, 32 and 64 vertices.}\label{fig:ContSearch}
\end{figure}

%Descrever a figura
\end{document}
